% Generated by Sphinx.
\def\sphinxdocclass{report}
\documentclass[letterpaper,10pt,english]{sphinxmanual}
\usepackage[utf8]{inputenc}
\DeclareUnicodeCharacter{00A0}{\nobreakspace}
\usepackage{cmap}
\usepackage[T1]{fontenc}
\usepackage{babel}
\usepackage{times}
\usepackage[Bjarne]{fncychap}
\usepackage{longtable}
\usepackage{sphinx}
\usepackage{multirow}

\addto\captionsenglish{\renewcommand{\figurename}{Fig. }}
\addto\captionsenglish{\renewcommand{\tablename}{Table }}
\floatname{literal-block}{Listing }



\title{Delta Stepping Algorithm in Parallel : A beginner approach. Documentation}
\date{May 16, 2016}
\release{0.0.1}
\author{Ivan Felipe Rodriguez Rodriguez}
\newcommand{\sphinxlogo}{}
\renewcommand{\releasename}{Release}
\makeindex

\makeatletter
\def\PYG@reset{\let\PYG@it=\relax \let\PYG@bf=\relax%
    \let\PYG@ul=\relax \let\PYG@tc=\relax%
    \let\PYG@bc=\relax \let\PYG@ff=\relax}
\def\PYG@tok#1{\csname PYG@tok@#1\endcsname}
\def\PYG@toks#1+{\ifx\relax#1\empty\else%
    \PYG@tok{#1}\expandafter\PYG@toks\fi}
\def\PYG@do#1{\PYG@bc{\PYG@tc{\PYG@ul{%
    \PYG@it{\PYG@bf{\PYG@ff{#1}}}}}}}
\def\PYG#1#2{\PYG@reset\PYG@toks#1+\relax+\PYG@do{#2}}

\expandafter\def\csname PYG@tok@gd\endcsname{\def\PYG@tc##1{\textcolor[rgb]{0.63,0.00,0.00}{##1}}}
\expandafter\def\csname PYG@tok@gu\endcsname{\let\PYG@bf=\textbf\def\PYG@tc##1{\textcolor[rgb]{0.50,0.00,0.50}{##1}}}
\expandafter\def\csname PYG@tok@gt\endcsname{\def\PYG@tc##1{\textcolor[rgb]{0.00,0.27,0.87}{##1}}}
\expandafter\def\csname PYG@tok@gs\endcsname{\let\PYG@bf=\textbf}
\expandafter\def\csname PYG@tok@gr\endcsname{\def\PYG@tc##1{\textcolor[rgb]{1.00,0.00,0.00}{##1}}}
\expandafter\def\csname PYG@tok@cm\endcsname{\let\PYG@it=\textit\def\PYG@tc##1{\textcolor[rgb]{0.25,0.50,0.56}{##1}}}
\expandafter\def\csname PYG@tok@vg\endcsname{\def\PYG@tc##1{\textcolor[rgb]{0.73,0.38,0.84}{##1}}}
\expandafter\def\csname PYG@tok@m\endcsname{\def\PYG@tc##1{\textcolor[rgb]{0.13,0.50,0.31}{##1}}}
\expandafter\def\csname PYG@tok@mh\endcsname{\def\PYG@tc##1{\textcolor[rgb]{0.13,0.50,0.31}{##1}}}
\expandafter\def\csname PYG@tok@cs\endcsname{\def\PYG@tc##1{\textcolor[rgb]{0.25,0.50,0.56}{##1}}\def\PYG@bc##1{\setlength{\fboxsep}{0pt}\colorbox[rgb]{1.00,0.94,0.94}{\strut ##1}}}
\expandafter\def\csname PYG@tok@ge\endcsname{\let\PYG@it=\textit}
\expandafter\def\csname PYG@tok@vc\endcsname{\def\PYG@tc##1{\textcolor[rgb]{0.73,0.38,0.84}{##1}}}
\expandafter\def\csname PYG@tok@il\endcsname{\def\PYG@tc##1{\textcolor[rgb]{0.13,0.50,0.31}{##1}}}
\expandafter\def\csname PYG@tok@go\endcsname{\def\PYG@tc##1{\textcolor[rgb]{0.20,0.20,0.20}{##1}}}
\expandafter\def\csname PYG@tok@cp\endcsname{\def\PYG@tc##1{\textcolor[rgb]{0.00,0.44,0.13}{##1}}}
\expandafter\def\csname PYG@tok@gi\endcsname{\def\PYG@tc##1{\textcolor[rgb]{0.00,0.63,0.00}{##1}}}
\expandafter\def\csname PYG@tok@gh\endcsname{\let\PYG@bf=\textbf\def\PYG@tc##1{\textcolor[rgb]{0.00,0.00,0.50}{##1}}}
\expandafter\def\csname PYG@tok@ni\endcsname{\let\PYG@bf=\textbf\def\PYG@tc##1{\textcolor[rgb]{0.84,0.33,0.22}{##1}}}
\expandafter\def\csname PYG@tok@nl\endcsname{\let\PYG@bf=\textbf\def\PYG@tc##1{\textcolor[rgb]{0.00,0.13,0.44}{##1}}}
\expandafter\def\csname PYG@tok@nn\endcsname{\let\PYG@bf=\textbf\def\PYG@tc##1{\textcolor[rgb]{0.05,0.52,0.71}{##1}}}
\expandafter\def\csname PYG@tok@no\endcsname{\def\PYG@tc##1{\textcolor[rgb]{0.38,0.68,0.84}{##1}}}
\expandafter\def\csname PYG@tok@na\endcsname{\def\PYG@tc##1{\textcolor[rgb]{0.25,0.44,0.63}{##1}}}
\expandafter\def\csname PYG@tok@nb\endcsname{\def\PYG@tc##1{\textcolor[rgb]{0.00,0.44,0.13}{##1}}}
\expandafter\def\csname PYG@tok@nc\endcsname{\let\PYG@bf=\textbf\def\PYG@tc##1{\textcolor[rgb]{0.05,0.52,0.71}{##1}}}
\expandafter\def\csname PYG@tok@nd\endcsname{\let\PYG@bf=\textbf\def\PYG@tc##1{\textcolor[rgb]{0.33,0.33,0.33}{##1}}}
\expandafter\def\csname PYG@tok@ne\endcsname{\def\PYG@tc##1{\textcolor[rgb]{0.00,0.44,0.13}{##1}}}
\expandafter\def\csname PYG@tok@nf\endcsname{\def\PYG@tc##1{\textcolor[rgb]{0.02,0.16,0.49}{##1}}}
\expandafter\def\csname PYG@tok@si\endcsname{\let\PYG@it=\textit\def\PYG@tc##1{\textcolor[rgb]{0.44,0.63,0.82}{##1}}}
\expandafter\def\csname PYG@tok@s2\endcsname{\def\PYG@tc##1{\textcolor[rgb]{0.25,0.44,0.63}{##1}}}
\expandafter\def\csname PYG@tok@vi\endcsname{\def\PYG@tc##1{\textcolor[rgb]{0.73,0.38,0.84}{##1}}}
\expandafter\def\csname PYG@tok@nt\endcsname{\let\PYG@bf=\textbf\def\PYG@tc##1{\textcolor[rgb]{0.02,0.16,0.45}{##1}}}
\expandafter\def\csname PYG@tok@nv\endcsname{\def\PYG@tc##1{\textcolor[rgb]{0.73,0.38,0.84}{##1}}}
\expandafter\def\csname PYG@tok@s1\endcsname{\def\PYG@tc##1{\textcolor[rgb]{0.25,0.44,0.63}{##1}}}
\expandafter\def\csname PYG@tok@gp\endcsname{\let\PYG@bf=\textbf\def\PYG@tc##1{\textcolor[rgb]{0.78,0.36,0.04}{##1}}}
\expandafter\def\csname PYG@tok@sh\endcsname{\def\PYG@tc##1{\textcolor[rgb]{0.25,0.44,0.63}{##1}}}
\expandafter\def\csname PYG@tok@ow\endcsname{\let\PYG@bf=\textbf\def\PYG@tc##1{\textcolor[rgb]{0.00,0.44,0.13}{##1}}}
\expandafter\def\csname PYG@tok@sx\endcsname{\def\PYG@tc##1{\textcolor[rgb]{0.78,0.36,0.04}{##1}}}
\expandafter\def\csname PYG@tok@bp\endcsname{\def\PYG@tc##1{\textcolor[rgb]{0.00,0.44,0.13}{##1}}}
\expandafter\def\csname PYG@tok@c1\endcsname{\let\PYG@it=\textit\def\PYG@tc##1{\textcolor[rgb]{0.25,0.50,0.56}{##1}}}
\expandafter\def\csname PYG@tok@kc\endcsname{\let\PYG@bf=\textbf\def\PYG@tc##1{\textcolor[rgb]{0.00,0.44,0.13}{##1}}}
\expandafter\def\csname PYG@tok@c\endcsname{\let\PYG@it=\textit\def\PYG@tc##1{\textcolor[rgb]{0.25,0.50,0.56}{##1}}}
\expandafter\def\csname PYG@tok@mf\endcsname{\def\PYG@tc##1{\textcolor[rgb]{0.13,0.50,0.31}{##1}}}
\expandafter\def\csname PYG@tok@err\endcsname{\def\PYG@bc##1{\setlength{\fboxsep}{0pt}\fcolorbox[rgb]{1.00,0.00,0.00}{1,1,1}{\strut ##1}}}
\expandafter\def\csname PYG@tok@mb\endcsname{\def\PYG@tc##1{\textcolor[rgb]{0.13,0.50,0.31}{##1}}}
\expandafter\def\csname PYG@tok@ss\endcsname{\def\PYG@tc##1{\textcolor[rgb]{0.32,0.47,0.09}{##1}}}
\expandafter\def\csname PYG@tok@sr\endcsname{\def\PYG@tc##1{\textcolor[rgb]{0.14,0.33,0.53}{##1}}}
\expandafter\def\csname PYG@tok@mo\endcsname{\def\PYG@tc##1{\textcolor[rgb]{0.13,0.50,0.31}{##1}}}
\expandafter\def\csname PYG@tok@kd\endcsname{\let\PYG@bf=\textbf\def\PYG@tc##1{\textcolor[rgb]{0.00,0.44,0.13}{##1}}}
\expandafter\def\csname PYG@tok@mi\endcsname{\def\PYG@tc##1{\textcolor[rgb]{0.13,0.50,0.31}{##1}}}
\expandafter\def\csname PYG@tok@kn\endcsname{\let\PYG@bf=\textbf\def\PYG@tc##1{\textcolor[rgb]{0.00,0.44,0.13}{##1}}}
\expandafter\def\csname PYG@tok@o\endcsname{\def\PYG@tc##1{\textcolor[rgb]{0.40,0.40,0.40}{##1}}}
\expandafter\def\csname PYG@tok@kr\endcsname{\let\PYG@bf=\textbf\def\PYG@tc##1{\textcolor[rgb]{0.00,0.44,0.13}{##1}}}
\expandafter\def\csname PYG@tok@s\endcsname{\def\PYG@tc##1{\textcolor[rgb]{0.25,0.44,0.63}{##1}}}
\expandafter\def\csname PYG@tok@kp\endcsname{\def\PYG@tc##1{\textcolor[rgb]{0.00,0.44,0.13}{##1}}}
\expandafter\def\csname PYG@tok@w\endcsname{\def\PYG@tc##1{\textcolor[rgb]{0.73,0.73,0.73}{##1}}}
\expandafter\def\csname PYG@tok@kt\endcsname{\def\PYG@tc##1{\textcolor[rgb]{0.56,0.13,0.00}{##1}}}
\expandafter\def\csname PYG@tok@sc\endcsname{\def\PYG@tc##1{\textcolor[rgb]{0.25,0.44,0.63}{##1}}}
\expandafter\def\csname PYG@tok@sb\endcsname{\def\PYG@tc##1{\textcolor[rgb]{0.25,0.44,0.63}{##1}}}
\expandafter\def\csname PYG@tok@k\endcsname{\let\PYG@bf=\textbf\def\PYG@tc##1{\textcolor[rgb]{0.00,0.44,0.13}{##1}}}
\expandafter\def\csname PYG@tok@se\endcsname{\let\PYG@bf=\textbf\def\PYG@tc##1{\textcolor[rgb]{0.25,0.44,0.63}{##1}}}
\expandafter\def\csname PYG@tok@sd\endcsname{\let\PYG@it=\textit\def\PYG@tc##1{\textcolor[rgb]{0.25,0.44,0.63}{##1}}}

\def\PYGZbs{\char`\\}
\def\PYGZus{\char`\_}
\def\PYGZob{\char`\{}
\def\PYGZcb{\char`\}}
\def\PYGZca{\char`\^}
\def\PYGZam{\char`\&}
\def\PYGZlt{\char`\<}
\def\PYGZgt{\char`\>}
\def\PYGZsh{\char`\#}
\def\PYGZpc{\char`\%}
\def\PYGZdl{\char`\$}
\def\PYGZhy{\char`\-}
\def\PYGZsq{\char`\'}
\def\PYGZdq{\char`\"}
\def\PYGZti{\char`\~}
% for compatibility with earlier versions
\def\PYGZat{@}
\def\PYGZlb{[}
\def\PYGZrb{]}
\makeatother

\renewcommand\PYGZsq{\textquotesingle}

\begin{document}

\maketitle
\tableofcontents
\phantomsection\label{index::doc}


Contents:


\chapter{Introduction}
\label{Introduction:introduction}\label{Introduction:welcome-to-delta-stepping-algorithm-in-parallel-a-beginner-approach-s-documentation}\label{Introduction::doc}\label{Introduction:id1}
Julia language is a high-level, high-performance dynamic programming language for technical computing. Between several other features  it provides a sophisticated compiler and distributed parallel execution, that allows the user to code sophisticated applications. Given the novelty of the language, is hard to find documentation that help the beginners and new learners to understand the core concepts and advantages of using Julia. For this reason, and mostly as a new learner, I provide this work that show by example some usages and advantages of using Julia .

The example I will be refering to is the Delta-stepping algorithm. Which is a clever proposition developed by  Madduri et. all. In their paper Parallel Shortest Path Algorihtm for Solving Large Scale Instances; along with the one by M. Kranjcevic \$Delta\$-Stepping Algorithm for Shared Memories Architectures.

This algorithm will be specially used  to find the contour of an image. Using the package \href{https://github.com/timholy/Images.jl}{Images} I converted the image to gray scale and calculate the gradient to compute a function cost. Later I  set a directed graph where the pixels where the nodes and the cost to go to a neighbor pixel with different color was high, then using the delta-stepping algorithm I can find the shortest path, which was also the contour.

A parallel implementation in Julia of this algorithm is propposed. For this implementation I ran several test to evaluate it's performance. In particular three images were selected.

There are many changes that still have to be done. In the next version, I will be updating the implementation changing the buckets from a dictionary of sets(not very efficient) to a shared list (to provide better parallelism). As well I will be working on the tutorials so that clear and nicer examples can be shown.


\section{A Little Bit of Julia.}
\label{Introduction:a-little-bit-of-julia}
Instalation might be very tricky. However for learning I suggest you to download the binary extension that can be found \href{http://julialang.org/downloads/}{Here}

When you have it go to the folder \emph{julia/bin} and simply execute \emph{Julia}. Like:

\begin{Verbatim}[commandchars=\\\{\}]
\PYG{n}{julia}
\end{Verbatim}



This initialization would start julia by default with one process. If you want to start with more processes you need to specify it with the following command (It makes sense when -p \# correlates with the number of processors on your machine). For example on mine two cores.

\begin{Verbatim}[commandchars=\\\{\}]
julia \PYGZhy{}p 2
\end{Verbatim}



In case you forget to start with this command, you can add more procs from the julia command line:

\begin{Verbatim}[commandchars=\\\{\}]
\PYG{n}{addprocs}\PYG{p}{(}\PYG{p}{)}
\end{Verbatim}



Which will add all the available processors by default, but you can specify too just puting inside the parenthesis as many process as cores has your machine.

You can get at any momment the number of available workers by running the following code:

\begin{Verbatim}[commandchars=\\\{\}]
\PYG{n}{nprocs}\PYG{p}{(}\PYG{p}{)}
\end{Verbatim}




\section{Search():}
\label{Introduction:search}


The function \code{search()} takes up to three arguments.
1. A ponymodel, the databse entity where you want to perform the search.
2. The \code{search\_string}, what  you are looking for; and,
3. The arguments, some additional options for more refined searching.

\begin{Verbatim}[commandchars=\\\{\}]
\PYG{n}{search}\PYG{p}{(}\PYG{n}{PonyModel}\PYG{p}{,} \PYG{l+s}{\PYGZdq{}}\PYG{l+s}{query}\PYG{l+s}{\PYGZdq{}}\PYG{p}{,} \PYG{o}{*}\PYG{o}{*}\PYG{n}{kw}\PYG{p}{)}
\end{Verbatim}

For example, if  you want  the results to be sorted by some specific searcheable field,
you have to indicate so, by adding the argument \code{sortedby="field"}.

In this case the search results object would show as a score the value of the item you choose for sorting. Please note that in order for
one field to be sortable, you must indicate it when you are registering
the model. (Refer to the \emph{Usage} section above)

\begin{Verbatim}[commandchars=\\\{\}]
\PYG{g+gp}{\PYGZgt{}\PYGZgt{}\PYGZgt{} }\PYG{k+kn}{from} \PYG{n+nn}{app} \PYG{k+kn}{import} \PYG{o}{*}
\PYG{g+gp}{\PYGZgt{}\PYGZgt{}\PYGZgt{} }\PYG{k+kn}{from} \PYG{n+nn}{flask\PYGZus{}ponywhoosh} \PYG{k+kn}{import} \PYG{n}{search}
\PYG{g+gp}{\PYGZgt{}\PYGZgt{}\PYGZgt{} }\PYG{n}{search}\PYG{p}{(}\PYG{n}{User}\PYG{p}{,}\PYG{l+s}{\PYGZdq{}}\PYG{l+s}{harol}\PYG{l+s}{\PYGZdq{}}\PYG{p}{,} \PYG{n}{sortedby}\PYG{o}{=}\PYG{l+s}{\PYGZdq{}}\PYG{l+s}{age}\PYG{l+s}{\PYGZdq{}}\PYG{p}{)}
\PYG{g+go}{\PYGZob{}\PYGZsq{}cant\PYGZus{}results\PYGZsq{}: 2,}
\PYG{g+go}{ \PYGZsq{}facet\PYGZus{}names\PYGZsq{}: [],}
\PYG{g+go}{ \PYGZsq{}matched\PYGZus{}terms\PYGZsq{}: \PYGZob{}\PYGZsq{}name\PYGZsq{}: [\PYGZsq{}felipe\PYGZsq{}]\PYGZcb{},}
\PYG{g+go}{ \PYGZsq{}results\PYGZsq{}: [\PYGZob{}\PYGZsq{}docnum\PYGZsq{}: 4L,}
\PYG{g+go}{              \PYGZsq{}rank\PYGZsq{}: 0,}
\PYG{g+go}{              \PYGZsq{}pk\PYGZsq{} : 5,,}
\PYG{g+go}{              \PYGZsq{}score\PYGZsq{}: \PYGZsq{}19\PYGZsq{}\PYGZcb{},}
\PYG{g+go}{             \PYGZob{}\PYGZsq{}docnum\PYGZsq{}: 11L,}
\PYG{g+go}{              \PYGZsq{}rank\PYGZsq{}: 1,}
\PYG{g+go}{              \PYGZsq{}pk\PYGZsq{} : 12,,}
\PYG{g+go}{              \PYGZsq{}score\PYGZsq{}: \PYGZsq{}19\PYGZsq{}\PYGZcb{}],}
\PYG{g+go}{ \PYGZsq{}runtime\PYGZsq{}: 0.0012810230255126953\PYGZcb{}}
\end{Verbatim}

In synthesis, the options available are: \code{sortedby}, \code{scored}, \code{limit}, \code{optimize}, \code{reverse}. Which are widely described in the whoosh documentation.


\section{The Attribute \texttt{\_pw\_.}}
\label{Introduction:the-attribute-pw}
There are some special features avalaible for models from the database. You just have to call the model \code{PonyModel.\_pw\_.}:
\begin{itemize}
\item {} 
\code{add\_field}: This function is to add a desired field in the index.

\item {} 
\code{charge\_documents}: This function let you charge an index from an  existing database.

\item {} 
\code{delete\_documents}: This function deletes all the documents stored in certain whoosh index.

\item {} 
\code{delete\_field}: This function works in case that you want to erase a determined field from a schema.

\item {} 
\code{update\_documents}: This function deletes all the documents and recharges them again.

\item {} 
\code{counts}: This function counts all the documents existing in an index.

\end{itemize}


\section{Searching by field:}
\label{Introduction:searching-by-field}
{\color{red}\bfseries{}\textbar{}byfield\textbar{}}
.. code:: python
\begin{quote}

search(PonyModel, query, field=''field\_name'')
\end{quote}

By default the function \code{search()} performs a multifield parser query, i.e.  you will be searching in all the fields you have declared when you registered the model. However, sometimes you would like to perform searching in just one or some of all the fields.
For these reasons we implemented the following extra options: The first one is refered as \code{field} all you have to do is indicate in which field you want to search. The output would be a results object containing only the information found in that field. And \code{fields} where you should write a list with all the fields you want to search.

\begin{Verbatim}[commandchars=\\\{\}]
\PYG{g+gp}{\PYGZgt{}\PYGZgt{}\PYGZgt{} }\PYG{n}{search}\PYG{p}{(}\PYG{n}{User}\PYG{p}{,}\PYG{l+s}{\PYGZdq{}}\PYG{l+s}{harol}\PYG{l+s}{\PYGZdq{}}\PYG{p}{,}\PYG{n}{field}\PYG{o}{=}\PYG{l+s}{\PYGZdq{}}\PYG{l+s}{name}\PYG{l+s}{\PYGZdq{}}\PYG{p}{)}
\PYG{g+go}{     \PYGZob{}\PYGZsq{}cant\PYGZus{}results\PYGZsq{}: 4,}
\PYG{g+go}{     \PYGZsq{}facet\PYGZus{}names\PYGZsq{}: [],}
\PYG{g+go}{     \PYGZsq{}matched\PYGZus{}terms\PYGZsq{}: \PYGZob{}\PYGZsq{}name\PYGZsq{}: [\PYGZsq{}harol\PYGZsq{}]\PYGZcb{},}
\PYG{g+go}{     \PYGZsq{}results\PYGZsq{}: [\PYGZob{}\PYGZsq{}docnum\PYGZsq{}: 1L,}
\PYG{g+go}{                  \PYGZsq{}pk\PYGZsq{}: u\PYGZsq{}7\PYGZsq{},}
\PYG{g+go}{                  \PYGZsq{}rank\PYGZsq{}: 0,}
\PYG{g+go}{                  \PYGZsq{}score\PYGZsq{}: 2.0296194171811583\PYGZcb{},}
\PYG{g+go}{                 \PYGZob{}\PYGZsq{}docnum\PYGZsq{}: 5L,}
\PYG{g+go}{                  \PYGZsq{}pk\PYGZsq{}: u\PYGZsq{}6\PYGZsq{},}
\PYG{g+go}{                  \PYGZsq{}rank\PYGZsq{}: 1,}
\PYG{g+go}{                  \PYGZsq{}score\PYGZsq{}: 2.0296194171811583\PYGZcb{},}
\PYG{g+go}{                 \PYGZob{}\PYGZsq{}docnum\PYGZsq{}: 12L,}
\PYG{g+go}{                  \PYGZsq{}pk\PYGZsq{}: u\PYGZsq{}13\PYGZsq{},}
\PYG{g+go}{                  \PYGZsq{}rank\PYGZsq{}: 2,}
\PYG{g+go}{                  \PYGZsq{}score\PYGZsq{}: 2.0296194171811583\PYGZcb{},}
\PYG{g+go}{                 \PYGZob{}\PYGZsq{}docnum\PYGZsq{}: 13L,}
\PYG{g+go}{                  \PYGZsq{}pk\PYGZsq{}: u\PYGZsq{}14\PYGZsq{},}
\PYG{g+go}{                  \PYGZsq{}rank\PYGZsq{}: 3,}
\PYG{g+go}{                  \PYGZsq{}score\PYGZsq{}: 2.0296194171811583\PYGZcb{}],}
\PYG{g+go}{     \PYGZsq{}runtime\PYGZsq{}: 0.005359172821044922\PYGZcb{}}

\PYG{g+gp}{\PYGZgt{}\PYGZgt{}\PYGZgt{} }\PYG{n}{search}\PYG{p}{(}\PYG{n}{Attribute}\PYG{p}{,}\PYG{l+s}{\PYGZdq{}}\PYG{l+s}{tejo}\PYG{l+s}{\PYGZdq{}}\PYG{p}{,} \PYG{n}{fields}\PYG{o}{=}\PYG{p}{[}\PYG{l+s}{\PYGZdq{}}\PYG{l+s}{sport}\PYG{l+s}{\PYGZdq{}}\PYG{p}{,}\PYG{l+s}{\PYGZdq{}}\PYG{l+s}{name}\PYG{l+s}{\PYGZdq{}}\PYG{p}{]}\PYG{p}{)}
\PYG{g+go}{    \PYGZob{}\PYGZsq{}cant\PYGZus{}results\PYGZsq{}: 4,}
\PYG{g+go}{     \PYGZsq{}facet\PYGZus{}names\PYGZsq{}: [],}
\PYG{g+go}{     \PYGZsq{}matched\PYGZus{}terms\PYGZsq{}: \PYGZob{}\PYGZsq{}name\PYGZsq{}: [\PYGZsq{}tejo\PYGZsq{}], \PYGZsq{}sport\PYGZsq{}: [\PYGZsq{}tejo\PYGZsq{}]\PYGZcb{},}
\PYG{g+go}{     \PYGZsq{}results\PYGZsq{}: [\PYGZob{}\PYGZsq{}docnum\PYGZsq{}: 1L,}
\PYG{g+go}{                  \PYGZsq{}pk\PYGZsq{}: u\PYGZsq{}7\PYGZsq{},}
\PYG{g+go}{                  \PYGZsq{}rank\PYGZsq{}: 0,}
\PYG{g+go}{                  \PYGZsq{}score\PYGZsq{}: 5.500610730717037\PYGZcb{},}
\PYG{g+go}{                 \PYGZob{}\PYGZsq{}docnum\PYGZsq{}: 6L,}
\PYG{g+go}{                  \PYGZsq{}pk\PYGZsq{}: u\PYGZsq{}1\PYGZsq{},}
\PYG{g+go}{                  \PYGZsq{}rank\PYGZsq{}: 1,}
\PYG{g+go}{                  \PYGZsq{}score\PYGZsq{}: 5.500610730717037\PYGZcb{}],}
\PYG{g+go}{     \PYGZsq{}runtime\PYGZsq{}: 0.006212949752807617\PYGZcb{}}
\end{Verbatim}


\section{add\_wildcards and something}
\label{Introduction:add-wildcards-and-something}
{\color{red}\bfseries{}\textbar{}wildcards\textbar{}}

\begin{Verbatim}[commandchars=\\\{\}]
\PYG{n}{search}\PYG{p}{(}\PYG{n}{PonyModel}\PYG{p}{,} \PYG{n}{query}\PYG{p}{,} \PYG{n}{add\PYGZus{}wildcards}\PYG{o}{=}\PYG{n+nb+bp}{True}\PYG{p}{)}
\end{Verbatim}

Whoosh  sets a wildcard \code{*},{}`{}`?{}`{}`,{}`{}`!{}`{}` by default to perform search for inexact terms, however sometimes  is desirable to search by exact terms instead. For this reason we added two more options: \code{add\_wildcards} and \code{something}.

The option \emph{add\_wildcards} (by default False)  is a boolean argument that tells the searcher whether it should or not include wild cards. For example, if you want to search ``harol'' when \code{add\_wildcards=False}, and you search by ``har'' the results would be 0. If \code{add\_wildcards=True} , then ``har'' would be fair enough to get the result ``harol''  because searching was performed  using wild cards.

\begin{Verbatim}[commandchars=\\\{\}]
\PYG{g+gp}{\PYGZgt{}\PYGZgt{}\PYGZgt{} }\PYG{n}{search}\PYG{p}{(}\PYG{n}{User}\PYG{p}{,} \PYG{l+s}{\PYGZdq{}}\PYG{l+s}{har}\PYG{l+s}{\PYGZdq{}}\PYG{p}{,} \PYG{n}{add\PYGZus{}wildcards}\PYG{o}{=}\PYG{n+nb+bp}{False}\PYG{p}{)}
\PYG{g+go}{  \PYGZob{}\PYGZsq{}cant\PYGZus{}results\PYGZsq{}: 0,}
\PYG{g+go}{   \PYGZsq{}facet\PYGZus{}names\PYGZsq{}: [],}
\PYG{g+go}{   \PYGZsq{}matched\PYGZus{}terms\PYGZsq{}: \PYGZob{}\PYGZcb{},}
\PYG{g+go}{   \PYGZsq{}results\PYGZsq{}: [],}
\PYG{g+go}{   \PYGZsq{}runtime\PYGZsq{}: 0.0003230571746826172}
\PYG{g+go}{   \PYGZcb{}}

\PYG{g+gp}{\PYGZgt{}\PYGZgt{}\PYGZgt{} }\PYG{n}{search}\PYG{p}{(}\PYG{n}{User}\PYG{p}{,} \PYG{l+s}{\PYGZdq{}}\PYG{l+s}{har}\PYG{l+s}{\PYGZdq{}}\PYG{p}{,} \PYG{n}{add\PYGZus{}wildcards}\PYG{o}{=}\PYG{n+nb+bp}{True}\PYG{p}{)}
\PYG{g+go}{  \PYGZob{}\PYGZsq{}cant\PYGZus{}results\PYGZsq{}: 4,}
\PYG{g+go}{   \PYGZsq{}facet\PYGZus{}names\PYGZsq{}: [],}
\PYG{g+go}{   \PYGZsq{}matched\PYGZus{}terms\PYGZsq{}: \PYGZob{}\PYGZsq{}name\PYGZsq{}: [\PYGZsq{}harol\PYGZsq{}]\PYGZcb{},}
\PYG{g+go}{   \PYGZsq{}results\PYGZsq{}: [\PYGZob{}\PYGZsq{}docnum\PYGZsq{}: 1L,}
\PYG{g+go}{                \PYGZsq{}pk\PYGZsq{}: u\PYGZsq{}7\PYGZsq{},}
\PYG{g+go}{                \PYGZsq{}rank\PYGZsq{}: 0,}
\PYG{g+go}{                \PYGZsq{}score\PYGZsq{}: 2.0296194171811583\PYGZcb{},}
\PYG{g+go}{               \PYGZob{}\PYGZsq{}docnum\PYGZsq{}: 5L,}
\PYG{g+go}{                \PYGZsq{}pk\PYGZsq{}: u\PYGZsq{}6\PYGZsq{},}
\PYG{g+go}{                \PYGZsq{}rank\PYGZsq{}: 1,}
\PYG{g+go}{                \PYGZsq{}score\PYGZsq{}: 2.0296194171811583\PYGZcb{},}
\PYG{g+go}{               \PYGZob{}\PYGZsq{}docnum\PYGZsq{}: 12L,}
\PYG{g+go}{                \PYGZsq{}pk\PYGZsq{}: u\PYGZsq{}13\PYGZsq{},}
\PYG{g+go}{                \PYGZsq{}rank\PYGZsq{}: 2,}
\PYG{g+go}{                \PYGZsq{}score\PYGZsq{}: 2.0296194171811583\PYGZcb{},}
\PYG{g+go}{               \PYGZob{}\PYGZsq{}docnum\PYGZsq{}: 13L,}
\PYG{g+go}{                \PYGZsq{}pk\PYGZsq{}: u\PYGZsq{}14\PYGZsq{},}
\PYG{g+go}{                \PYGZsq{}rank\PYGZsq{}: 3,}
\PYG{g+go}{                \PYGZsq{}score\PYGZsq{}: 2.0296194171811583\PYGZcb{}],}
\PYG{g+go}{   \PYGZsq{}runtime\PYGZsq{}: 0.014926910400390625\PYGZcb{}}
\end{Verbatim}

The \code{something=True} option, would run first a search with
\code{add\_wildcards=False} value, but in case results are empty it would automatically run a search adding wildcards to the result.

\begin{Verbatim}[commandchars=\\\{\}]
\PYG{g+gp}{\PYGZgt{}\PYGZgt{}\PYGZgt{} }\PYG{n}{search}\PYG{p}{(}\PYG{n}{Attribute}\PYG{p}{,} \PYG{l+s}{\PYGZdq{}}\PYG{l+s}{tejo}\PYG{l+s}{\PYGZdq{}}\PYG{p}{,} \PYG{n}{something} \PYG{o}{=} \PYG{n+nb+bp}{True}\PYG{p}{)}
\PYG{g+go}{  \PYGZob{}\PYGZsq{}cant\PYGZus{}results\PYGZsq{}: 4,}
\PYG{g+go}{   \PYGZsq{}facet\PYGZus{}names\PYGZsq{}: [],}
\PYG{g+go}{   \PYGZsq{}matched\PYGZus{}terms\PYGZsq{}: \PYGZob{}\PYGZsq{}name\PYGZsq{}: [\PYGZsq{}tejo\PYGZsq{}], \PYGZsq{}sport\PYGZsq{}: [\PYGZsq{}tejo\PYGZsq{}]\PYGZcb{},}
\PYG{g+go}{   \PYGZsq{}results\PYGZsq{}: [\PYGZob{}\PYGZsq{}docnum\PYGZsq{}: 1L,}
\PYG{g+go}{                \PYGZsq{}pk\PYGZsq{}: u\PYGZsq{}7\PYGZsq{},}
\PYG{g+go}{                \PYGZsq{}rank\PYGZsq{}: 0,}
\PYG{g+go}{                \PYGZsq{}score\PYGZsq{}: 5.500610730717037\PYGZcb{},}
\PYG{g+go}{               \PYGZob{}\PYGZsq{}docnum\PYGZsq{}: 6L,}
\PYG{g+go}{                \PYGZsq{}pk\PYGZsq{}: u\PYGZsq{}1\PYGZsq{},}
\PYG{g+go}{                \PYGZsq{}rank\PYGZsq{}: 1,}
\PYG{g+go}{                \PYGZsq{}score\PYGZsq{}: 5.500610730717037\PYGZcb{}],}
\PYG{g+go}{   \PYGZsq{}runtime\PYGZsq{}: 0.0036530494689941406\PYGZcb{}}
\end{Verbatim}


\section{The output dictionary}
\label{Introduction:the-output-dictionary}
The \code{search()} function returns a dictionary with selected information.
\begin{itemize}
\item {} 
\code{cant\_results}: is the total number of documents collected by the searcher.

\item {} 
\code{facet\_names}: is useful with the option \code{groupedby}, because it returns the item used to group the results.

\item {} 
\code{matched\_terms}: is a dictionary that saves the searcheable field and the match given by the query.

\item {} 
\code{runtime}: how much time the searcher took to find it.

\item {} 
\code{results}: is  a dictionary's list for the individual results. i.e. a dictionary for every single result, containing:
\begin{itemize}
\item {} 
`rank': the position of the result,

\item {} 
`result': indicating the primary key and the correspond value of the item,

\item {} 
`score': the score for the item in the search, and

\item {} 
`pk': the primary key Or the sets of primary keys.

\end{itemize}

\end{itemize}


\section{use\_dict:}
\label{Introduction:use-dict}

\chapter{Indices and tables}
\label{index:indices-and-tables}\begin{itemize}
\item {} 
\DUspan{xref,std,std-ref}{genindex}

\item {} 
\DUspan{xref,std,std-ref}{modindex}

\item {} 
\DUspan{xref,std,std-ref}{search}

\end{itemize}



\renewcommand{\indexname}{Index}
\printindex
\end{document}
